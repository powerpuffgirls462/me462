\documentclass[11pt]{article}
\usepackage[turkish]{babel}
\usepackage[utf8]{inputenc}
\usepackage{graphicx}
\usepackage{textcomp}
\usepackage{fixltx2e}
\usepackage{setspace}
\usepackage{amsmath}
\usepackage{parskip}
\usepackage{fancyhdr}
\usepackage{vmargin}
\usepackage{comment}
\usepackage{multirow}
\usepackage{mathrsfs}
\begin{document}
\textbf{{\Huge Karakterler ve Senaryoları}}
\\\\\\
{\Large 1)Göz karakteri}
\\\\
\textbf{\textit{Olması gerekenler:}}\\
\begin{itemize}
\item Bu karakterin 2 gözü olmalıdır .\\ \item Gözleriyle parmak takibi veya yüz takibi yapmalıdır.\\ 

\item Üzerindeki ya da masadaki kamera Buğra hocanın masası ve en ön sırayı algılamalıdır.\\
\item Tekli göz kapağı olmalıdır. 
Bu kapak ile göz açma kapama hareketini yapmalıdır.\\
\end{itemize}
\textbf{\textit{Olursa karakteri iyi duruma getirenler:}}\\
\begin{itemize}
\item Gözlere ek olarak kaşlar da eklenebilir.\\
\item Boyun eklenerek baş kısmının dönüş hareketi gerçekleştirilebilir.\\
Agız eklenebilir.\\
\item Kamera ile tüm sınıf algılanabilir.\\
\item Büyüyüp küçülen iris mekanizması eklenebilir.\\
\item Tekli göz kapağı yerine ikili göz kapağı yapılabilir.\\\\
\end{itemize}
\textbf{\textit{Uyanma senaryoları:}}\\
\begin{itemize}
\item Kamera ile derinlik algılatarak karaketerimize yaklaşan cisim varsa göz kapaklarını açarak uyanır.\\
\item Kişiler dokunursa dokunma, sensörlerle algılatılıp karakter uyanır.\\
\item Labımızda ses seviyesi yüksek olduğunda uyan{\i}r.\\
\item İsmini duyduğunda ya da 'Wake up' söylenildiğinde uyanır.\\
\item Işık ile uyanır.\\\\
\end{itemize}
\textbf{\textit{Takip etme senaryoları:}}\\
\begin{itemize}
\item Yeni bir kişi laba girdiğinde ona doğru dönüp gözleri onu takip eder.\\
\item Masada Buğra hocanın gözlerini takip eder.\\
\item Diğer karakterin veya kişilerin parmakları yaklaşınca kaşları ve gözleri ile parmağı takip eder.\\\\
\end{itemize}
\textbf{\textit{Duygu, tepki senaryoları:}}\\
\begin{itemize}
\item \c{C}ok yüksek seslerde kaşlarını çatarak kızar.\\
\item Kameranın görüş açısı kapanırsa yine kaşları ile kızma duygusunu verir.\\
\item Karaktere çok yakın bir cisim yaklaştığında, gözünü kapatır veya kafasını geriye doğru hareket ettirir.
\item İsmini duyduğunda kaşlarını kaldırıp, gözleri büyür.\\
\item Voice recognition ile Buğra hocanın sesini algılar ve mutlu olur, gözlerini kırpar ve o yöne gözlerini çevirir.\\
\item Laba giren kişiyi fark ettiğinde kişi uzaktayken irisin küçülmesi, yakınlaştığında büyümesi hareketi gözlenir.\\\\
\end{itemize}
\textbf{\textit{Uyuma senaryoları:}}\\
\begin{itemize}
\item Işık kapalı olduğunda, göz kapakları kapanarak uyur.\\
\item Labda ışık açıkken uyuma saatinde uyur. Uyuma saatinden 1 saat önce uykusunun geldiğini hissettirmek için gözleri küçülür.\\
\item Uyanma saatinden önce uyandırılırsa 5-10 dakika içerisinde geri uyur.\\\\\\
\end{itemize}
{\Large 2)Kukla karakteri}
\\\\
\textbf{\textit{Olması gerekenler:}}\\
\begin{itemize}
\item Kol, bacak ve eklem hareketlerini yapmalıdır.\\
\item Önünde şapkası vardır.\\\\
\end{itemize}
\textbf{\textit{Senaryolar:}}\\
\begin{itemize}
\item Buğra hoca bir kişiyi işaret ettiğinde karakter de elini hareket ettirerek o kişiyi işaret eder.\\
\item Lab çok gürültülü iken parmağını ağzına götürerek 'sus' işareti yapar.\\
\item Elleri ve kafasını sallayarak yeni gelen kişilere selam verir.\\
\item Sınıfa bir kişi geldiğinde ellerini ve ayaklarını hareket ettirerek dans eder.\\
\item Önündeki şapkasının içine pul konduğunda önce dans eder sonra şapkayı ayağı ile itekleyerek şapkanın içinden pulun düşmesini ve kişinin pulu geri almasını sağlamış olur.\\
\item Kapı ile haberleşilip sınıfa Buuğra hocanın girdiği bilgisi alınıp sınıfta ses var mı yok mu kontrol edilerek dersin başlamış olduğu bilgisi edinilir. Bu bilgi ile karakter dersi anladım hissini vermek için kafasını yukarı aşağı sallar.\\
\item Sesin hangi yönden geldiği algılatarak kafasını o yöne çevirir.\\
\item Belli hareket çeşitleri olup onları tekrar eder.\\
\item Karşısındaki kişiye 'çak' yapar.\\
\item Buğra hoca derse başladığında ve bitirdiğinde karakterdeki butona basmalıdır. Ders başladıktan sonra 50 dakika geçtiğinde butona daha basılmadıysa ellerini hareket ettirerek dikkat çeker. 50 dakikadan sonra butona basılmadıysa her 10 dakikada bir bu hareketi tekrar eder.\\
\item Buğra hocanın hareketlerini taklit eder.\\
\item Işık kapalı olduğunda, uyuma saati geldiğinde kafasını sağa düşürerek uyur.\\
\item Yaklaşan bir cisim olduğunda, karaktere dokunulduğunda, ışıklı ve sesli ortamlarda gerinme hareketlerini yaparak uyanır.\\\\
\end{itemize}
{\Large 3)Oyuncu karakteri}
\\\\
\textbf{\textit{Senaryolar:}}\
\begin{itemize}
\item Karakterin yanına bir kişi yaklaştığında 'Do you want to play a game?' cümlesini söyler. Kişi 'evet' derse oyun oynamaya başlarlar. Buna ek olarak bir kişinin 'Let's play a game.' dediğinde de oyun oynarlar.\\
\item Oyuna başlarken karakter kişiye hangi oyun oynamak istediğini sorar. Karakter 4 çeşit oyun oynayabilmektedir: Hangman, Tic-Tac-Toe, Battleship, Minefield. Kişi bu oyunlardan bir tanesini seçer.\\
\item Oyunu gözlerinden lazer yayarak ya da dokunmatik ekran üzerinde kolunu kaldırıp ekrana tıklayarak oynar.\\
\item Kazanınca kahkaha atar, kaybedince 'I want to play again.' der.\\\\
\end{itemize}
{\Large 4)Ses karakteri}
\\\\
\textbf{\textit{Senaryolar:}}\
\begin{itemize}
\item Ses takibi yapar. Sesin hangi yönden geldiğini fark ederek o yöne kafasını çevirir.\\
\item Belirli kelimelere ya da seslere spesifik özelliği vardır. Örneğin, iyi akşamlar dendiğinde 'Görüşürüz' der.\\
\item Lab sessizken karakter kendi rutin hareketini yapar. Örneğin, tavuk karakteri sürekli tekrarladığı yem bulmak amaçlı kafasını oynatma hareketini yapar.\\\\
\end{itemize}
{\Large 5)İlgiye muhtaç karakter}
\\\\
\textbf{\textit{Senaryolar:}}\
\begin{itemize}
\item İnsanlar hareket halinde iken ilgi çekmek amaçlı, insanların karakteri fark edip ona gitmesini sağlamak için dikkat çekici hareketler yapar. Örneğin, el ve ayaklarını hareket ettirebilir, renk değiştirebilir, ses çıkarabilir. \\\\
\end{itemize}
\end{document}
